\textbf{Búvárruha } &  Olyan eszköz, amivel a vízbe vagy lyukba esett szereplő ezt az eseményt túléli, nem hal meg. \\[10px]
\textbf{Elásott tárgy } &  Olyan tárgy, ami jégtáblába van fagyva, azaz nem szedhetőek fel, nem használhatóak. Ha ezeket kiássa egy szereplő, a tárgy eszközként hozzá kerül.  \\[10px]
\textbf{Élelem } &  Olyan eszköz, ami 1-gyel növeli az őt felvevő szereplő testhőjét.  \\[10px]
\textbf{Eszközhasználat } &  Minden eszköz tud valamit csinálni, ez automatikusan végrehajtásra kerül szükség esetén.  \\[10px]
\textbf{Eszköz } &  Használható, funkcióval rendelkező dolog. Lásd részletesen: Lapát, kötél, búvárruha, élelem.  \\[10px]
\textbf{Halál } &  Egy szereplő halála a játék végét, veszteséget jelent. Társasjáték analógiával élve játékos kiesése. Mivel a játék kooperatív, így egy játékos kiesése a teljes csapat kiesését, és így a játék elvesztését jelenti.  \\[10px]
\textbf{Hó } &  A jégtáblákat borítja, a szereplők mozgását nem akadályozza, viszont elásott tárgy csak akkor látható meg, ha az adott jégtáblán már nincsen hó.  \\[10px]
\textbf{Hóásás } &  Egy réteg hó eltakarítása, azaz törlése az adott jégtábláról. Egy egység munkába kerül.  \\[10px]
\textbf{Hóréteg } &  Egy munkavégzéssel eltakarítható hó. Egy jégtáblán több hóréteg lehet.  \\[10px]
\textbf{Hóvihar } &  Egy véletlenszerűen bekövetkező esemény, ami néhány jégtáblát érint. Az érintett jégtáblákon álló szereplők egy testhőt veszítenek, kivéve, ha igluban vannak. Ezeken a jégtáblákon néhány réteg új hó is keletkezik.  \\[10px]
\textbf{Iglu } &  Az eszkimó, az egyik szerepkör képességeként építhető, a szereplő aktuális jégtáblára rakható módosító. Bárki, aki az adott jégtáblára lép, automatikusan az igluba kerül, ahonnan a hóviharok átvészelhetőek.  \\[10px]
\textbf{Instabil jégtábla } &  Van egy maximum létszáma, ahány szereplő állhat rajta, mielőtt a tábla átfordul, és a szereplők a vízbe esnek.  \\[10px]
\textbf{Játékos } &  A szereplőt irányító ember.  \\[10px]
\textbf{Játéktábla } &  A társasjátékokban használatos jelentést használjuk, azaz a teljes játéktér. Szinonima a jégmezőre.   \\[10px]
\textbf{Jégmező } &  A szereplők ezen mozoghatnak. Társasjáték analógiával élve játéktábla.  \\[10px]
\textbf{Jégtábla } &  A jégmező egy darabja. Társasjáték analógiával élve a tábla egy mezője.  \\[10px]
\textbf{Jelzőfény } &  Jelzőrakéta alkatrész, eszköz.  \\[10px]
\textbf{Jelzőrakéta } &  Pisztoly, jelzőfény, és patron egy jégtáblára hordásával, és szereplőnként egy egységnyi munkával összerakható eszköz. Ennek összerakása a játék végét és megnyerését jelenti.  \\[10px]
\textbf{Képességhasználat } &  A szereplő a tulajdonságainak megfelelő adottságot használja. \\[10px]
\textbf{Képességhasználat } &  Mindegyik szerepkör végre tud hajtani egy speciális dolgot, ami valamilyen módon segíti a szereplőket. Ezeknek a használata munkavégzés, így nem automatikus, a játékos dönthet úgy, hogy ezt használja.  \\[10px]
\textbf{Kör } &  A játék menetének egysége. Egy körben minden játékos sorra kerül, egymás után, egyesével.  \\[10px]
\textbf{Kötél } &  Olyan eszköz, amivel lyukba vagy vízbe esett szereplő kimenthető, ekkor a kimentett szereplő arra a jégtáblára kerül, amelyen az a szereplő állt, aki a kötéllel kimentette a másikat.  \\[10px]
\textbf{Lapát } &  Olyan eszköz, amivel hóásásnál egy egység munkával két réteg havat is el lehet takarítani.  \\[10px]
\textbf{Mező } &  A társasjátékokban használatos jelentést használjuk, azaz a játéktábla egy része. Szinonima a jégtáblára.  \\[10px]
\textbf{Mozgás } &  A szereplő egyik jégtábláról egy másik jégtáblára való átlépése. Társasjáték analógiával élve bábu lépése.  \\[10px]
\textbf{Mozog } &  Azon folyamat amikor a szereplő a pályán egy szomszédos jégtáblára lép. \\[10px]
\textbf{Munkaegység } &  A munkavégzés költsége.  \\[10px]
\textbf{Munkavégzés } &  Egységekben végezhető tevékenység. Ilyen a mozgás, hóásás, eszközhasználat, és a képességhasználat.  \\[10px]
\textbf{Patron } &  Jelzőrakéta alkatrész, eszköz.  \\[10px]
\textbf{Pisztoly } &  Jelzőrakéta alkatrész, eszköz.  \\[10px]
\textbf{Stabil jégtábla } &  Bárhány szereplő állhat rajta.  \\[10px]
\textbf{Szereplő } &  Irányítható figura. Társasjáték analógiával élve bábu.  \\[10px]
\textbf{Tenger } &  A játéktábla határait jelöli ki, nem lehet belemenni.  \\[10px]
\textbf{Testhő } &  Videojáték analógiával életpont. Egy egyszerű számláló. Mindegyik szereplő adott értékkel kezd, bizonyos események ezt növelik, mások pedig csökkentik. Ha a számláló leér 0-ra, a szereplő meghal.  \\[10px]
\textbf{Vízbe/lyukba esés } &  ha a jégtábla lyukas vagy a kapacitása már nem elég, a játékos(ok) vízbe kerülnek. \\[10px]
