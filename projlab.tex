%--------------------------------------------------------------------------------------
% ÁLTALÁNOS ALAPSABLON A (RÉSZ)BEADANDÓKHOZ
%--------------------------------------------------------------------------------------
\documentclass[12pt, a4paper]{book}
\usepackage[margin=1in]{geometry}
\usepackage[utf8]{inputenc}
\usepackage[magyar]{babel}
\usepackage[hidelinks]{hyperref} 
\usepackage{graphicx}
\usepackage{comment}
\usepackage{amsthm}
\usepackage{MnSymbol}
\usepackage{wasysym}
\usepackage{fancyhdr}
\usepackage{longtable}
\usepackage{changepage}
\usepackage{xcolor}
\usepackage{algorithmicx}
\usepackage{listings}

\pagestyle{fancy}
\fancyhf{}
\fancyhead[LE,RO]{\thepage}
\fancyhead[RE,LO]{\leftmark}
%kitöltendő
\fancyfoot[CE,CO]{Csapatsorszám - Csapatnév}

\title{Szoftver projekt labor beadandó} 
%kitöltendő
\author{Karsa Zoltán, Killin Attila, Jakab Dániel, Kacsó Péter, Imets Tamás} 

\setcounter{secnumdepth}{4}

\begin{document}
\begin{titlepage} 
	\begin{center}
		\includegraphics[width=5cm]{doc/img/BMElogo}\\ 
		\vspace{1cm}
		\LARGE{\bfseries{Szoftver projekt laboratórium}}\\
		\vspace{0.5cm}
		%kitöltendő
		\Large\textsc{X. Összesített dokumentum}\\
		\vspace{1cm}
		\small{Csapat}\\
		%kitöltendő
		\Large{\bfseries{Csapatsorszám - Csapatnév}}\\
		\vspace{1cm}
		\small{Konzulens}\\
		%kitöltendő
		\Large{Dr. Goldschmidt Balázs}
	\end{center}
	\begin{flushleft}
		\vspace*{6cm}
		Csapattagok\\
		\vspace{-0.3cm}
		\rule{14cm}{0.5pt}\\
		\vspace{0.2cm}
		\begin{tabular}{l l l} %kitöltendő
			Teszt Tamás & ?????? & teszttamas@iit.bme.hu \\
			Java Dániel & ?????? & javadaniel@iit.bme.hu \\
			Projlab Péter & ?????? & projlabpeter@iit.bme.hu \\
			LaTex Zoltán  & ?????? & latexzoltan@iit.bme.hu \\
			Git Attila & ?????? & gitattila@iit.bme.hu \\
		\end{tabular}
	\end{flushleft}
	\begin{flushright}
		\vspace*{2cm}
		\today   %vagy saját dátum megadása, manuálisan
	\end{flushright}
\end{titlepage}
\input{definitions}
\sloppy

%-----------------------------------------------------------------------------------------

%Végső beadáshoz tartalomjegyzék:
%\tableofcontents

%A fejezetek számozásának átállítására (mindig a megadott számtól folytatódik a számozás):
%\setcounter{chapter}{-1}


\setcounter{chapter}{1}

\input{doc/02doc}
\clearpage
\input{doc/02naplo}

\input{doc/03doc}
\clearpage
\input{doc/03naplo}

\input{doc/04doc}
\clearpage
\input{doc/04naplo}

\input{doc/05doc}
\clearpage
\input{doc/05naplo}

\input{doc/06doc}
\clearpage
\input{doc/06naplo}

\input{doc/07doc}
\clearpage
\input{doc/07naplo}

\input{doc/08doc}
\clearpage
\input{doc/08naplo}

\setcounter{chapter}{9}
\input{doc/10doc}
\clearpage
\input{doc/10naplo}

\input{doc/11doc}
\clearpage
\input{doc/11naplo}

\setcounter{chapter}{12}
\input{doc/13doc}
\clearpage
\input{doc/13naplo}

\input{doc/14doc}

\end{document}